% Amath 582 HW 1
%%%%%%%%%%%%%%%%%%%%%%%%%%%%%%%%%%%%%%%%%
% Stylish Article
% LaTeX Template
% Version 2.1 (1/10/15)
%
% This template has been downloaded from:
% http://www.LaTeXTemplates.com
%
% Original author:
% Mathias Legrand (legrand.mathias@gmail.com) 
% With extensive modifications by:
% Vel (vel@latextemplates.com)
%
% License:
% CC BY-NC-SA 3.0 (http://creativecommons.org/licenses/by-nc-sa/3.0/)
%
%%%%%%%%%%%%%%%%%%%%%%%%%%%%%%%%%%%%%%%%%

%----------------------------------------------------------------------------------------
%	PACKAGES AND OTHER DOCUMENT CONFIGURATIONS
%----------------------------------------------------------------------------------------

\documentclass[fleqn,10pt]{../SelfArx} % Document font size and equations flushed left

\usepackage[english]{babel} % Specify a different language here - english by default

\usepackage{amsmath}
\usepackage{bm}			% For bold symbols
\usepackage{graphicx}	% For images and figures

%----------------------------------------------------------------------------------------
%	COLUMNS
%----------------------------------------------------------------------------------------

\setlength{\columnsep}{0.55cm} % Distance between the two columns of text
\setlength{\fboxrule}{0.75pt} % Width of the border around the abstract

%----------------------------------------------------------------------------------------
%	COLORS
%----------------------------------------------------------------------------------------

\definecolor{color1}{RGB}{0,0,90} % Color of the article title and sections
\definecolor{color2}{RGB}{0,20,20} % Color of the boxes behind the abstract and headings

%----------------------------------------------------------------------------------------
%	HYPERLINKS
%----------------------------------------------------------------------------------------

\usepackage{hyperref} % Required for hyperlinks
\hypersetup{hidelinks,colorlinks,breaklinks=true,urlcolor=color2,citecolor=color1,linkcolor=color1,bookmarksopen=false,pdftitle={Title},pdfauthor={Author}}

%----------------------------------------------------------------------------------------
%	ARTICLE INFORMATION
%----------------------------------------------------------------------------------------

\JournalInfo{Amath 582 HW 2} % Journal information
\Archive{Brian de Silva} % Additional notes (e.g. copyright, DOI, review/research article)

\PaperTitle{Amath 582 Homework 2 \\ \small{Due: January 28, 2016}} % Article title

\Authors{Brian de Silva\textsuperscript{1}} % Authors
\affiliation{\textsuperscript{1}\textit{Department of Applied Mathematics, University of Washington, Seattle}} % Author affiliation

\Keywords{} % Keywords - if you don't want any simply remove all the text between the curly brackets
\newcommand{\keywordname}{Keywords} % Defines the keywords heading name

%----------------------------------------------------------------------------------------
%	ABSTRACT
%----------------------------------------------------------------------------------------

\Abstract{Math math math, mathy mathy math}

%----------------------------------------------------------------------------------------

\begin{document}

\flushbottom % Makes all text pages the same height

\maketitle % Print the title and abstract box

\tableofcontents % Print the contents section

% \thispagestyle{empty} % Removes page numbering from the first page

%----------------------------------------------------------------------------------------
%	ARTICLE CONTENTS
%----------------------------------------------------------------------------------------

\section{Introduction}
\label{sec:introduction}
This assignment deals with time-frequency analysis and has two parts. In Part I we analyze the frequencies present at different times in a segment of Handel's Messiah. We use this as an opportunity to experiment with various Gabor windows and the parameters associated with them such as window width and time step length. We explore the spectrograms generated by the various parameter choices. We then provide some analysis of our findings.

In Part II we are given two recordings of the hit song \textit{Mary had a little lamb}, one played on the piano and the other on the recorder. We use our newfound expertise to reconstruct the music score for each audio file using just their spectrograms (and a table of the frequencies corresponding to each musical note). We then compare and contrast the spectrograms of the song performed on the two instruments.

This writeup is structured as follows: in Section \ref{sec:theoretical_background} we give a brief review of the theory behind our techniques. Next we discuss our algorithms and their implementations in MATLAB in Section \ref{sec:algorithms_implementation_and_development} before discussing our numerical results in Section \ref{sec:computational_results}. Finally we make some closing remarks in Section \ref{sec:summary_and_conclusions}.

% section introduction (end)


\section{Theoretical Background} % (fold)
\label{sec:theoretical_background}

\subsection{Gabor transforms}
The Fast Fourier Transform (FFT) is an indispensible tool for analyzing the frequency content of a given signal However it has at least one major drawback, it identifies all the frequencies present in a data set simultaneously, without giving any information about \textit{when} they occured, i.e. when taking an FFT we lose all temporal information (or spatial information if we interpret the independent variable as being space-like). Gabor transforms are an attempt to remedy this issue. The novel idea behind Gabor transforms is simply to make a slight adjustment to the kernel used in Fourier Transforms so that only local (in time) information is transformed into the frequency domain. By varying a parameter in the transform one can shift around the instant in time from which one wishes to extract frequency information. By varying another one can specify how wide a range of times is considered ``local''.

The Fourier Transform is given by
\begin{equation}
	\int^\infty_{-\infty}f(\tau)e^{-ik\tau}d\tau = (f,e^{ikt}).
\end{equation}
Note that it may be expressed as an inner product between $f$ and $e^{ik\tau}$. Given the kernel
\begin{equation}
	g_{t,k}(\tau) = e^{ik\tau}g(\tau-t)
\end{equation}
we define the \textit{Gabor transform} of an $L^1$ function $f$ as the inner product between $f$ and $g_{t,k}$:
\begin{equation}
	\tilde f_g(t,k) = \int^\infty_{-\infty}f(\tau)\bar g(\tau-t)e^{-ik\tau}d\tau = (f,g_{t,k}).
\end{equation}
The function $g(\tau-t)$ is some function which is typically very small outside some radius of 0, $a$, of $0$. Its purpose is to create a window of time outside of which the values of $f$ are severely damped, and inside of which its output is relatively unchanged. For example, in this assignment we experiment with the following window functions:
\begin{align}
	g(t) &= e^{-at^2} &\quad \mathrm{(Gaussian)} \\
	g(t) &= e^{-at^{10}} &\quad \mathrm{(Super-Gaussian)} \\
	g(t) &= \left(1-\left(\frac{t}{a}\right)^2\right)e^{-\left(\frac{t}{\sqrt{2}a}\right)^2} &\quad \mathrm{(Mexican~Hat)} \\
	g(t) &= 
				\begin{cases}
					0 & |t|>1/2 \\
					1 & |t|\leq 1/2
				\end{cases}	&\quad \mathrm{(Step~function)}.
\end{align}

There is a tradeoff which is in some sense inherent in using Gabor transforms. We are able to retain some temporal information when moving into the frequency domain, but the range of frequencies which we can resolve decreases. This is because using a window which does not encompass the entire temporal domain means that very long frequencies present in the signal will not be captured within the window and therefore will not appear in the spectrogram. Taking a window that is large enough to resolve these frequencies means sacrificing temporal precision and aiming for extreme temporal resolution by taking a very small window results in the loss of the ability to detect all but the shortest frequencies. These tradeoffs are explored more fully in Section \ref{sec:algorithms_implementation_and_development_part1}.

Throughout this section we have mentioned spectrograms without having defined them. To construct one we first discretize the temporal domain into suitably small subintervals by selecting equally spaced points in time. For each of these points we apply a Gabor filter centered at the point to the original signal, take a FFT of the filtered signal, and stack the resulting frequency profiles in a 2-D array. We then use the MATLAB function \texttt{pcolor} to plot the absolute value of the resulting array. This plot is the spectrogram. It gives one an idea of which frequencies are prominent in the signal at each of the selected points in time. Some examples are shown in \ref{sec:computational_results}.

% section theoretical_background (end)


\section{Algorithms Implementation and Development} % (fold)
\label{sec:algorithms_implementation_and_development}
In this section we describe the numerical algorithms developed to explore various parameters associated with Gabor transforms and to determine the scores of the piano and recorder sound clips. The procedures are given below along with a few comments and pertinent details.

\subsection{Part I}
\label{sec:algorithms_implementation_and_development_part1}
Below we outline the MATLAB script written to allow experimentation with over and undersampling, various Gabor windows, and Gabor window widths.

\subsection*{1. Read in the audio file and take its DFT}
This gives us an idea of the frequency components present in the unmodified data. Note that all temporal information is lost once the FFT is taken. As before we must scale the wavenumbers by $2\pi/L$ since MATLAB's \texttt{fft2} function assumes its input to come from a $2\pi$-periodic function. Here $L$ is the length of the interval of interest. Since the audio file produces a vector with an odd number of entries, in order to produce the right number of frequency components, we identify the first and last entries of the vector. The FFT assumes the data to be $2\pi$-periodic so our actions are somewhat justified. It turns out that the first and last entries are both zero in any case.

\subsection*{2. Create the desired Gabor window}
As detailed in Section \ref{sec:theoretical_background}, we test out a standard Gaussian filter, a super-Gaussian, the Mexican Hat wavelet, and a step function. At this stage we also specify the width of the window as well as the time discretization spacing, $dt$.

\subsection*{3. Create a spectrogram by evolving time in increments of $dt$}
At each time step we apply the window chosen in step 2 to the data, take the  FFT of the filtered data, then store the magnitude of the result in a row of the 2-dimensional spectogram array.

\subsection*{4. Plot the spectrogram}

\subsection{Part II}
Below we sketch the alrogithm used to reconstruct the music score for the two audio clips of \textit{Mary had a little lamb} and to compare the spectrograms of the two.

\subsection*{1. Read in the audio files and take their DFTs}
This step mimicks that of the first step in Section \ref{sec:algorithms_implementation_and_development_part1} except the sound vectors are both even in length so no modification is necessary.

\subsection*{2. Create the Gabor window}
Here we use the standard Gaussian Gabor window introduced in lecture which is given explicity in Section \ref{sec:theoretical_background}. We also specify the width of the window and the time discretization spacing $dt$. We had success using the same parameters for both of the two sound files in Part II.

\subsection*{3. Create spectrograms by evolving time in increments of $dt$}
This step is the same as in Part I.

\subsection*{4. Plot the spectrograms}
% section algorithm_implementation_and_development (end)


\section{Computational Results} % (fold)
\label{sec:computational_results}

\subsection{Part I}
Here are all my Part I pics:

\subsection{Part II}

Here are all my Part II pics:
% \begin{figure}
% \centering
% \includegraphics[width=\linewidth]{recorder_signal}
% \caption{Top: plot of sound clip of Mary had a little lamb played on the recorder. Bottom: Magnitude of the FFT of the sound data}
% \label{fig:recorder_signal}
% \end{figure}

% \begin{figure}
% \centering
% \includegraphics[width=\linewidth]{piano_signal}
% \caption{Top: plot of sound clip of Mary had a little lamb played on the piano. Bottom: Magnitude of the FFT of the sound data}
% \label{fig:piano_signal}
% \end{figure}

\begin{figure}
\centering
\includegraphics[width=\linewidth]{recorder_gabor}
\caption{Top: plot of sound clip of Mary had a little lamb played on the recorder and the Gabor filter at one instant in time. Middle: Gabor filtered signal. Bottom: Magnitude of the Gabor transform of the sound data corresponding to the above filter.}
\label{fig:recorder_gabor}
\end{figure}

\begin{figure}
\centering
\includegraphics[width=\linewidth]{piano_gabor}
\caption{Top: plot of sound clip of Mary had a little lamb played on the piano and the Gabor filter at one instant in time. Middle: Gabor filtered signal. Bottom: Magnitude of the Gabor transform of the sound data corresponding to the above filter.}
\label{fig:piano_gabor}
\end{figure}

\begin{figure}
\centering
\includegraphics[width=\linewidth]{recorder_spec}
\caption{A plot of the log of the spectrogram for the recorder signal (100 timesteps, width = 100).}
\label{fig:recorder_spec}
\end{figure}

\begin{figure}
\centering
\includegraphics[width=\linewidth]{recorder_spec_zoom}
\caption{A zoomed in plot of the log of the spectrogram for the recorder signal (100 timesteps, width = 100).}
\label{fig:recorder_spec_zoomed}
\end{figure}

\begin{figure}
\centering
\includegraphics[width=\linewidth]{piano_spec}
\caption{A plot of the log of the spectrogram for the piano signal (100 timesteps, width = 100).}
\label{fig:piano_spec}
\end{figure}

\begin{figure}
\centering
\includegraphics[width=\linewidth]{piano_spec_zoom}
\caption{A zoomed in plot of the log of the spectrogram for the piano signal (100 timesteps, width = 100).}
\label{fig:piano_spec_zoomed}
\end{figure}



% section computational_results (end)

\section{Summary and Conclusions} % (fold)
\label{sec:summary_and_conclusions}


% section summary_and_conclusions (end)




%------------------------------------------------
\phantomsection
\section*{Appendix A: MATLAB Functions} % The \section*{} command stops section numbering
Here we outline the nonstandard MATLAB functions used to complete this assignment.
\vskip 0.5cm

\noindent \texttt{\textbf{audioread('filename')}}: This function reads the sound file specified by \texttt{filename} and returns two outputs: a vector of sampled sound data and the sample rate for said data.
\vskip 0.3cm

\noindent \texttt{\textbf{audioplayer(y,Fs)}}: This command creates an audioplayer object which can play the sound stored in the vector \texttt{y} with sampling rate \texttt{Fs}. It is typically paired with the \texttt{playblocking} function which actually plays the sound.
\vskip 0.3cm

\noindent \texttt{\textbf{fft(X)}}: Given a $1-$dimensional array of data, \texttt{X}, \texttt{fft2(X)} returns the $1-$dimensional DFT of \texttt{X}. The output has the same size as the input. To plot the frequency components given by \texttt{fft} one should first apply the \texttt{fftshift} command as \texttt{fft} returns a shifted version of the frequency components.
\vskip 0.3cm

\noindent \texttt{\textbf{fftshift(v)}}: This command shifts the vector/matrix output of \texttt{fft}, \texttt{fft2}, and \texttt{fftn} so that the 0 frequency lies at the center of the vector/matrix. For example, in one dimension \texttt{v = fft(data)} is a vector and \texttt{fftshift(v)} swaps the first and second halves of \texttt{v}. This command is useful for plotting the FFT of data.
\vskip 0.3cm

\noindent \texttt{\textbf{pcolor(X,Y,vals)}}: Plots the scalars stored in the 2-D array \texttt{vals} at the points specified by the 2-D arrays \texttt{X} and \texttt{Y} by mapping them to colors. \texttt{X}, \texttt{Y}, and \texttt{vals} should have the same dimensions. \texttt{X} and \texttt{Y} can be created with the \texttt{meshgrid} command.
\vskip 0.3cm

\noindent \texttt{\textbf{playblocking(AP)}}: \texttt{playblocking} takes an audioplayer object, \texttt{AP} as an input and plays the associated sound clip.
\vskip 0.3cm



\addcontentsline{toc}{section}{Appendix A} % Adds this section to the table of contents

\phantomsection
\section*{Appendix B: MATLAB Code}
See the following pages published in MATLAB for the implementation of the algorithm presented in Section \ref{sec:algorithms_implementation_and_development}.

\addcontentsline{toc}{section}{Appendix B}



%----------------------------------------------------------------------------------------
%	REFERENCE LIST
%----------------------------------------------------------------------------------------
% \phantomsection
% \bibliographystyle{unsrt}
% \bibliography{sample}

%----------------------------------------------------------------------------------------

\end{document}